\section{Conclusion}
We have presented two novel approaches to optimize the memory performance and SM utilization for depthwise and pointwise convolutions respectively. 
Our approach improves the data locality for convolutional operations performed on the row and column directions to reduce the memory access. 
Our techniques utilize the common GPU shuffle operations supported by mainstream GPU programming models, including CUDA and OpenCL, and do not require hardware modifications.
For pointwise convolution, the main problem is low SM utilization because cuDNN uses a fixed block strategy for all pointwise convolutions. We design a dynamic block strategy and meanwhile hide the memory access latency. 
We evaluate our approach on an NVIDIA RTX 2080Ti GPU platform. 
We compare our approach against a wide range of heavily optimized convolution algorithms. 
Experimental results show that our approach consistently outperforms the competing methods by delivering the best overall performance for the three types of
convolution tasks.


%For the 2D convolution, our approach outperforms the state-of-the-art image processing libraries. For the depth-wise convolution, our
%techniques deliver up to $4 \times$ speedups over the fastest algorithm of cuDNN. For the multi-channel 2D convolution, proposed reuse
%algorithms achieve an average speedup of $1.23\times$ over the fastest algorithm of cuDNN.


%\begin{acks}                            %% acks environment is optional
%                                        %% contents suppressed with 'anonymous'
%  %% Commands \grantsponsor{<sponsorID>}{<name>}{<url>} and
%  %% \grantnum[<url>]{<sponsorID>}{<number>} should be used to
%  %% acknowledge financial support and will be used by metadata
%  %% extraction tools.
%  This material is based upon work supported by the
%  \grantsponsor{GS100000001}{National Science
%    Foundation}{http://dx.doi.org/10.13039/100000001} under Grant
%  No.~\grantnum{GS100000001}{nnnnnnn} and Grant
%  No.~\grantnum{GS100000001}{mmmmmmm}.  Any opinions, findings, and
%  conclusions or recommendations expressed in this material are those
%  of the author and do not necessarily reflect the views of the
%  National Science Foundation.
%\end{acks}
