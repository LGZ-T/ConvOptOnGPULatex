\section{Related Work}
Over the past few years, many efforts have been made on optimizing convolution operations. Nowadays, GEMM-based, FFT-based and
Winograd-based convolutions are broadly adopted convolution algorithms.

GEMM-based convolution is the first attempt to optimize convolution. Chellapilla et al. \cite{Chellapilla2006High} developed an unrolling
convolution algorithm which is also called im2col convolution algorithm. They lower the convolutions into a matrix multiplication which is
highly optimized on GPU. Though there are some duplicate input elements involved in the computation, the performance gain is still
impressive. Chetlur et al. \cite{ChetlurWVCTCS14} at NVIDIA integrated the unrolling convolution algorithm into cuDNN.

As general matrix multiplication has been well optimized, researchers shift the focus from GEMM-based convolution to FFT-based and
Winograd-based convolutions. Both FFT-based and Winograd-based convolutions can reduce the computational complexity and improve the
performance of convolution. Mathieu et al. \cite{mathieu2013fast} proposed FFT-based convolution, they compute convolutions as pointwise
products in the Fourier domain and reusing transformed input data many times which significantly reduce the complexity of convolution. This
method greatly improves the performance of convolution compared with GEMM-based implementation. However, FFT-based convolution is more
suitable for large filters, because they need to pad filters to the same size as input data and small filters like $3 \times 3$ need more
memory than large filters. To overcome this drawback, Lavin et al. \cite{lavin2016fast} used Winograd’s minimal filtering algorithm to
accelerate convolution on GPU. Their algorithm can reduce the arithmetic complexity of convolution by up to a factor of 4 compared with
direct convolution. But Winograd algorithm is suitable for small filters due to its numerical instability. \cite{Zhen2018Optimizing}
extends Winograd algorithm to any filter sizes. However, both Winograd-based algorithms need to transform the input and filter before
performing matrix multiplication, and need more operations than FFT algorithm.

All abovementioned convolutions need to transform the input and filter before performing matrix multiplication, this will incur much memory
overhead and offset some performance gains brought by the reduction in computational complexity. Recent studies mainly focus on how to
reduce the memory overhead of transformation phases. \cite{cho2017mec} reduces the memory overhead of GEMM-based convolution. They use a
compact lowering scheme to reduce the redundancy in the lowered matrix and then do multiple small matrix multiplications in parallel.
However, their algorithm still needs to transform the input and filter tensors into lowered matrixes to compute convolution.
\cite{Iandola2014Communication} reduces memory  communication of 2D convolution on GPU. They prefetch image regions to registers, and let
each  thread do more work with fewer threads. Their method just uses registers to accelerate convolution and does not reduce the number of
memory accesses. Different from \cite{Iandola2014Communication}, we further optimize the usage of registers and significantly reduce the
number of memory accesses.
